\documentclass[english,a4paper,12pt]{refrep}
\settextfraction{1}
\usepackage[utf8]{luainputenc}
\usepackage{xcolor}
\usepackage{booktabs}
\usepackage{longtable}
\usepackage{tabularx}
\usepackage{rotating}
\usepackage{multicol}
\usepackage{multirow}
\usepackage[disable]{todonotes}
\usepackage{tocloft}
\usepackage{hyperref}
\usepackage{gensymb}
\makeatletter

% Font selection
\renewcommand{\familydefault}{\sfdefault}
\fontfamily{phv}\selectfont

% Reference colors
\hypersetup{
  unicode=true,
  linkcolor=blue, % internal links
  linktocpage=true, % only page numbers
  urlcolor=blue, % external links
  citecolor=green,
  filecolor=magenta,
  bookmarks=true,
  bookmarksnumbered=true,
  breaklinks=true, % wrap links is Ok
  colorlinks=true,
  pdftoolbar=true,
  pdfmenubar=true,
  pdfnewwindow=true
}

% Some shortcuts
\newcommand\xc{\textsf{XCSoar}}
\newcommand\fl{\textsf{Flarm}}
\newcommand\al{\textsf{Altair}}

% Define command to insert XCSoar website
\newcommand{\xcsoarwebsite}[1]{\url{https://xcsoar.org#1}}
\newcommand{\xcsoarforum}[1]{\url{https://forum.xcsoar.org#1}}

\input{figures.tex}

% Potentially overdue ``InfoBox'' style macro
\newcommand{\InfoBox}[0]{{InfoBox}}

% Enumerated todo's for the todonotes package
\newcounter{todocounter}
\newcommand{\todonum}[2][]{\stepcounter{todocounter}\todo[#1]{\thetodocounter: #2}}

\maxipagerulefalse

% Include XCSoar header and footer settings
\input{buttons.sty}
\widowpenalty=1000
\clubpenalty=1000

% the command \version prints the XCSoar version number
\newcommand{\version}{\begingroup\catcode`\_=\active\input{VERSION.txt}\endgroup}

% Define command to put a menu label on the margin
% aligned left
\newcommand{\menulabel}[1]{\marginpar{\parbox{5.0cm}{\raggedright #1}}}
% aligned right
\newcommand{\menulabelr}[1]{\marginpar{\parbox{4.05cm}{\raggedleft #1}}}

% Define some colors
\definecolor{AirspaceYellow}{rgb}{.99,.99,.19}
\definecolor{AirspaceRed}{rgb}{.99,.19,.19}

\usepackage{wasysym}
\usepackage{graphicx}        % -> includegraphics
\usepackage{paralist}
\hypersetup{
    pdftitle={XCSoar in a Flash},
    pdfauthor={WZ},
    pdfsubject={Quickstart},
    pdfproducer={xcsoar.org},
    pdfkeywords={XCSoar}{Quickstart}{Soaring}{Final Glide},
}
\parindent0mm                %no indent
\textwidth=170mm             %match print and screen
\setlength{\baselineskip}{1em}

% Set the page title
\input{xcsoar-headers.sty}
\xcsoarheader{XCSoar in a flash}

%
%##############  Colour, Background, Border ##############
%
\fboxrule0.4mm               % width Fbox-border
\definecolor{flashyellow}{rgb}{.9,.85,0}
\definecolor{flashblue}{rgb}{0,0,1}
%
%##############  advanced Macros and Menus ..##############
%
\newcommand{\nav}[3]{\bmenut{Nav}{#1/2}{\LARGE$\triangleright$}~\bmenut{#2}{#3}}%NavMenu
\newcommand{\display}[3]{\bmenut{Display}{#1/2}{\LARGE$\triangleright$}~\bmenut{#2}{#3}}%DisplayMenu
\renewcommand{\config}[3]{\bmenut{Config}{#1/3}{\LARGE$\triangleright$}~\bmenut{#2}{#3}}%ConfigNavMenu
\newcommand{\info}[3]{\bmenut{Info}{#1/3}{\LARGE$\triangleright$}~\bmenut{#2}{#3}}%InfoMenu
%
%
%##############  Document start of flash ##############
%
\setcounter{secnumdepth}{1}
\renewcommand*{\thesection}{\textsf{\arabic{section}}}
\begin{document}
%
%print-testchart 
%\tiny
%    \definecolor{mauve}{RGB}{224 176 255}
%    \begin{testcolors}[rgb,RGB,cmyk,hsb,HSB,HTML,gray,Gray]
%    \testcolor{black}
%    \testcolor{white}
%    \testcolor{darkgray}
%    \testcolor{gray}
%    \testcolor{lightgray}
%    \testcolor{red}
%    \testcolor{green}
%    \testcolor{blue}
%    \testcolor{cyan}
%    \testcolor{magenta}
%    \testcolor{yellow}
%    \testcolor{brown}
%    \testcolor{lime}
%    \testcolor{olive}
%    \testcolor{orange}
%    \testcolor{pink}
%    \testcolor{purple}
%    \testcolor{teal}
%    \testcolor{violet}
%    \testcolor{mauve}
%    \testcolor[cmyk]{0 0.5 1 0.42}
%    \testcolor{flashblue}
%    \testcolor{flashyellow}
%    \end{testcolors}
%\newpage
%
%
%##############  Titlepage of flash ##############
\thispagestyle{empty}
\begin{center}
\fontsize{34}{0}
\selectfont\textbf{XCSoar in a flash}
\fontsize{12}{12}
\vspace{0.1em}
\includegraphics[angle=0,width=0.58\linewidth,keepaspectratio='true']{figures/blitzlogo.png}

\vspace{1em}
{\Huge How To\\}
\vspace{0.2em}
{\tiny get your personal\\}
{\Huge Tactical Soaring\\ Assistant\\}
\vspace{0.2em}
{\tiny ready for boarding}
\end{center}

%##############  XCSoar Checklist of flash ##############
\newpage
%%%%%--------------------------------------------------------------------%%%%%
%%%%%--------------------------------------------------------------------%%%%%
%%%%%--------------------------------------------------------------------%%%%%
%%%%%--------------------------------------------------------------------%%%%%
%%%%%--------------------------------------------------------------------%%%%%
\begin{center}
{\Huge\textbf{XCSoar Checklist}}
\end{center}
\section*{Flash \textcolor{flashyellow}{On}}\label{ch:flashon}
%%%%%--------------------------------------------------------------------%%%%%

\subsection{\textcolor{flashblue}{Bring XCSoar into play}}
\begin{compactitem}
\item get hardware and install XCSoar
\item get appropriate data files of your flight district
\item tell XCSoar which data files to use
\item tell XCSoar about the glider's polar \& weight
\item possibly connect to instruments
\item finish setup and familiarize
\item mount hardware
\item add items listed underneath to your checklists
\item make ``home''
\end{compactitem}

\subsection*{\textcolor{flashblue}{Conduct preflight check, including}}
\begin{compactitem}
\item setup polar and weight
\item setup wind and flight parameters (MC, bugs, QNH)
\item possibly setup a task
\end{compactitem}

\subsection*{\textcolor{flashblue}{Conduct start check, including}}
\begin{compactitem}
\item Check wind and flight setup once more
\end{compactitem}

\subsection*{\textcolor{flashblue}{Fly, enjoy}}

\subsection*{\textcolor{flashblue}{Conduct after flight check}}
\begin{compactitem}
\item Download flight logs from logger, upload to skylines and OLC
\item Gather statistical data of flight.
\end{compactitem}

\section*{{\color[rgb]{.9,.85,0}Flash} Off}

\vspace{2em}
\hspace*{1cm} Flashing time too short?\\
\hspace*{4cm} Just a second...\\
\hspace*{6cm} flashing time set up to match,\\
\hspace*{10cm} now proceed:


%##############  Start of main body of flash ##############
\setlength{\parskip}{0.3\baselineskip}
\newpage
%%%%%--------------------------------------------------------------------%%%%%
%%%%%--------------------------------------------------------------------%%%%%
%%%%%--------------------------------------------------------------------%%%%%
%%%%%--------------------------------------------------------------------%%%%%
%%%%%--------------------------------------------------------------------%%%%%
\section{XCSoar Installation}\label{ch:XCSinstall}
%%%%%--------------------------------------------------------------------%%%%%

%\begin{wrapfigure}[10]{R}{0.23\textwidth}
%\vspace{-5.5em}
%  \begin{flushright}
%    \includegraphics[width=0.17\textwidth]{figures/xy01.jpg}
%    XY, ready for installation
%  \end{flushright}
%\end{wrapfigure}

Identify the appropriate distribution of XCSoar for your hardware and 
operating system \textsf{(Android, PC, Linux)} 
on \xcsoarwebsite{}.
In case you choose an install directory other than standard, make sure access 
rights are set to full/all. Create subdirectory \texttt{XCSoarData} and put all 
data files there, which are configuration, airspace, waypoints, and the XCM 
terrain files.

Using the default from installers for Android, Windows Mobile, PC will do the 
job smoothly.  In order to keep things easy, stay with the defaults. 
Installation on SD-card is possible.

\begin{compactitem}
\item[1.] Download latest XCSoar version released from homepage 
{\xcsoarwebsite{}} or via Google Play app for Android.
\item[2.] Download terrain file \texttt{name.xcm} from 
{\xcsoarwebsite{/download/maps/}}
corresponding to your flight district.  If \texttt{.xcm}-file appears to be 
corrupt, use another browser. (This is an issue with several versions of 
internet explorer, disintegrating \texttt{.xcm}-files.)
\item[3.] Download waypoints with XCSoar supporting several file formats 
\texttt{.dat, .cup, .wpz, .wpt}.  However, follow the roads starting with 
{\xcsoarwebsite{/download/data.html}}.
\item[4.] Download an airspace file \texttt{.txt} and/or \texttt{.sua}.  Again, follow
the link last mentioned.
\end{compactitem}

\subsection{\textcolor{flashblue}{Run XCSoar for the first time}}
Run XCSoar. On startup, XCSoar asks you to enter either a real life (Fly) or
a simulation (Sim) mode. In Sim mode, you can input speed, heading, and height 
manually in order to do some exercising on the ground. Initial position in Sim mode
can be set using task actions menu \bmenuw{Set As New Home}.
In Fly mode, instrument input of already connected devices are fed to XCSoar.
Remember this when  attempting to connect to instruments. In Sim mode, all 
connections are cut off.

Start fly mode. Tell XCSoar which files to use. Invoke system setup via main 
menu \button{Config}.

\begin{flushleft}\hspace*{1cm}\config{2}{System}{}\blink\bmenuw{Site Files}
\blink\bmenuw{Site files}\\\end{flushleft}
\index{Map database}\index{Waypoints}\index{Airspaces}

Choose appropriate files in dialogue. If no files show up, again make sure files 
reside in subdirectory \texttt{/XCSoarData}.

\subsection{\textcolor{flashblue}{File Manager}}\index{Filemanager}
\textbf{Please note:} XCSoar's File Manager on Android requires Android 2.2
and up!!!

If the platform you choose supports direct access to the internet, you can 
easily download and maintain data files directly. Touch 

\begin{flushleft}\hspace*{1cm}\config{2}{File}{Manager}\\\end{flushleft}
A list of already downloaded files appears.  Doing this for the very first 
time might result in displaying an empty list. Proceed with \blink Add.

A huge list of files shows up. If not, check your internet connection. The XCSoar
team maintains this list of files and corresponding URL, read in after
touching \blink Add in order to give you a plain list of possible downloads 
worldwide. (Take notice, that file names of equal files might differ to names, 
downloaded via other ways.)
Selecting files will start downloading immediately. To add more files, repeat 
\blink Add.
Updating files is done via \blink Download on the same menu screen.  As a 
quick start, download terrain, airspace and waypoint files apiece only. With 
some maps, e.g.\ depicting thermal spots and areas it might take some time to
understand setups. For further info please refer to the main manual or 
\url{http://www.glidinghotspots.eu/} 

\subsection{\textcolor{flashblue}{Language (Fonts, Sounds)}}
Typically, XCSoar follows the operation system's language setting. If you
prefer another, there are 28 languages available as of Dec. 2013 (sorry
Bavarian, Boston and other people, dialects are not to be included ever):

\begin{flushleft}\hspace*{1cm}\config{2}{System}{}\blink\bmenuw{Look}\blink
\bmenuw{Language, Input}\\\end{flushleft}

This menu page also allows for changing fonts and, after having prepared a
special data file, do much more by defining events.  However, this is not
supposed to be the scope of a quick start manual, just an outlook.

%%%%%--------------------------------------------------------------------%%%%%
%%%%%--------------------------------------------------------------------%%%%%
%%%%%--------------------------------------------------------------------%%%%%
%%%%%--------------------------------------------------------------------%%%%%
%%%%%--------------------------------------------------------------------%%%%%
\newpage
\section{Finish Configuration Setup}
%%%%%--------------------------------------------------------------------%%%%%

\subsection{\textcolor{flashblue}{Setup Polar}}

%\begin{wrapfigure}[7]{R}{0.22\textwidth}
%\vspace{-9em}
%  \begin{flushright}
%    \includegraphics[width=0.27\textwidth]{figures/xy02.png}
%    xy figuring out the missing colours of life without configuration.
%  \end{flushright}
%\end{wrapfigure}

XCSoar comes with a comprehensive list of predefined polars.  To choose, edit, 
or store your customized polar, invoke the plane configuration by 
\begin{flushleft}\hspace*{1cm}\config{2}{Plane}{}\\\end{flushleft}
The planes' indexes in this compilation are taken from the official DAEC list 
of 2012.

\subsection{\textcolor{flashblue}{Connect Logger, FLARM, Vario etc\dots NMEA setup}}
To run, XCSoar needs live data from a GPS receiver at least. You have to tell 
XCSoar, which one to use, even built-in hardware. To connect to, use
\begin{flushleft}\hspace*{1cm}\config{2}{Devices}{}\blink\bmenuw{edit}\\\end{flushleft}
Many portable devices are equipped with such hardware.  However, it is always 
a good idea to connect to one or more external devices:

\begin{compactitem}
\item A specialized GPS receiver gains much better reception providing much 
better data for measures and calculations.
\item An airspeed indicator allows quick and exact wind estimates without 
circling.
\item A vario improves the thermal assistant.
\item A task can be declared to an IGC logger, and after landing, the 
flight log can be downloaded.
\item Some varios allow synchronising the MacCready setting with XCSoar.
\item FLARM (and even ADS-B input) provide status information of others
around you - whereas XCSoar does NOT provide collision detection.
\end{compactitem}

\subsection{\textcolor{flashblue}{NMEA}}
Wondering?  NMEA stands for `National Marine Electronics Association', having 
reached the status of a worldwide standard. The standard defines electrical
interfaces as well as protocols used for transmitting navigational data and 
much more. As with nautical mile, aviators adopted maritime things once
more. So do their instruments...

\subsection{\textcolor{flashblue}{Internal GPS}}
The GPS receiver of your particular mobile device might be a really good one. 
For several reasons carefully check, whether this holds true in-flight:

\begin{compactitem}
\item The three-dimensional orientation of your device and in-built GPS 
antenna respectively may cause poor reception.  When losing GPS-fix, many
mobile devices are heavily dependent on aGPS - assisted GPS.  You have to wait 
minutes for a new fix in the air.
\item aGPS is based on information transmitted from a cellular network 
operator. Reception might get worse in the air and so does precision of the 
GPS fix.
\item Cheap or badly setup hardware `invents' GPS fixes. So-called 
`interpolated' and possibly bad values might screw up XCSoar's algorithms.
\end{compactitem}

If your hardware's GPS reception turns out to be stable: congratulations! You 
acquired a good piece of hardware. No need for extra hardware, you already own 
an XCSoar `basic' system. If not, the method of choice is to connect an extra 
GPS receiver. Better you connect an IGC logger. Best, you connect to a FLARM.
With an IGC Logger, you can declare tasks and download IGC-certified flight 
logs needed to participate in the OLC league.
Connecting FLARM will give you status information of your team mates,
traffic info and a barometric height reading - and IGC-certified flight 
logs if hardware is enabled.
With a FLARM connected you own an XCSoar `classic' system.  Integrating even
more hardware is a task of its own and not within scope of a quick start guide. 

\subsection{\textcolor{flashblue}{Glueing things together}}
Currently, there are three ways to connect to other instruments.  Use either
a serial wired, a wired serial-over-USB or radio connection via Bluetooth. The
latter being the method of choice with Android mobiles, still a so-called
Bluetooth Serial Port Profile (SPP) is needed. With version Android 4 and up,
Google introduced serious problems stemming from the serial Bluetooth profile
involved. With XCSoar coming for free, it is almost no risk to just test your 
hardware's capabilities. If everything is fine ;-)

\textbf{Please note:} Before connecting instruments via Bluetooth you have to
pair Bluetooth devices. Needed steps differ depending on the hardware you use.
Use your operating system's tools.

For using USB-equipped hardware, you need a serial-to-USB converter. Consult 
XCSoar's website for further details: {\xcsoarwebsite{/hardware/}}

If using a serial connection, with PDAs expected to be equipped with, be 
aware, there are three physically differing serial types of connection 
commonly used. It is about using different physical signal levels:

\begin{compactitem}
\item RS232 uses full-swing \pm 12 Volt level
\item TTL uses 5 Volt signal swing
\item Low level logic uses 3 Volt signal swing
\end{compactitem}
Logically speaking these connections all do the same, exactly the
same.

Make sure serial ports signal swing match.  Otherwise you need a level
converter. Again, consult XCSoar's website for more details.

Things go easier by using approved hardware setups. A widely used Android
hardware for running XCSoar is Dell's Streak 5. On the hardware page of 
XCSoar's website you find several Bluetooth adapters listed, known for
gaining stable radio connections. 

\subsection{\textcolor{flashblue}{Pilot's Name}}
\begin{flushleft}\hspace*{1cm}\config{2}{System}{}\blink\bmenuw{Setup}\blink\bmenuw{Logger}\\\end{flushleft}

\subsection{\textcolor{flashblue}{Time settings}}
\begin{flushleft}\hspace*{1cm}\config{2}{System}{}\blink\bmenuw{Setup}\blink\bmenuw{Time}\\\end{flushleft}

\subsection{\textcolor{flashblue}{Moving the (Moving) Map manually}}
Time to engage the map.  Navigate through by entering the pan mode.
\begin{flushleft}\hspace*{1cm}\display{1}{Pan}{On}\\\end{flushleft}
Depending on your hardware use buttons, dials, rockers of your hardware to 
navigate, or even use your fingertip for dragging the map.

\subsection{\textcolor{flashblue}{Make ``home''}}\index{home waypoint}
Whenever XCSoar starts up, the map display is centred on your ``home'' 
waypoint. Unless you set up your first ``true'' task, XCSoar's route function will 
show you the way home. If you do not even define home, XCSoar will create a 
waypoint automatically on location where it detects the takeoff condition - and 
show you the way back.
\begin{compactitem}
\item Navigate to desired waypoint
\item Click/tap on (touch) screen in close proximity of desired waypoint.
\item Dialogue ``Map Elements at this location'' appears.
\item Choose waypoint.
\item Push \bmenuw{Details}.
\item Turn one page further by touching: \blink
\item Push \bmenuw{Set As New Home}.
\end{compactitem}

With finishing the basic setup, you are almost ready to take one step further 
by proceeding with the preflight check. Otherwise, all of a sudden you might 
face a deep interest in getting familiar with all of the knobs and buttons 
XCSoar provides, or...

Possibly all of XCSoar's features start making you see X's everywhere? There
are two strategies to cope with this:

\begin{compactitem}
\item Change to XCSoar's main manual and take your time.
\item Do the last steps underneath and replay a flight log, just watching 
XCSoar's screen to familiarize with readings and things happening - or invoke 
XCSoar's simulation mode.
\end{compactitem}

...with the best strategy of all going through both.

\subsection{\textcolor{flashblue}{Action - watch XCSoar replay a flight}}
Pick a flight log and let XCSoar step back in time.  Place the file in 
subdirectory \verb+/XCSoarData/logs+ and start replay.

\begin{flushleft}\hspace*{1cm}\config{2}{Replay}{}\blink\bmenuw{File\dots Start}\\\end{flushleft}

Now you are able to study XCSoar's behaviour on the ground. Whilst exploring
you may develop a desire to change the screen layout and InfoBox
content. You are kindly requested to go through the following steps
only if you have developed a deep demand to do so.  XCSoar comes with a
quite useful pre-configuration.

\subsection*{\textcolor{flashblue}{Screen Layout (portrait, landscape, etc\dots)}}
To change basic Screen Layouts:
\begin{flushleft}\hspace*{1cm}\config{2}{System}{}\blink\bmenuw{Look}\blink
\bmenuw{Screen Layout}\\\end{flushleft}

\subsection*{\textcolor{flashblue}{Change InfoBox Content}}
To change InfoBox sets and boxes' content:
\begin{flushleft}\hspace*{1cm}\config{2}{System}{}\blink\bmenuw{Look}\blink
\bmenuw{InfoBox Sets}\\\end{flushleft}

\subsection{\textcolor{flashblue}{Configuration Files / Backup}}
For using XCSoar in different situations, environments, and gliders, you are 
invited to use profiles. If XCSoar finds more than one on startup, 
it will ask you which one to use. Here we are, recalling your profile is just a 
button to touch.
Thus, once happy with having configured almost everything, do a backup of your 
profile file. Other reasons are briefly depicted as follows.
\begin{compactitem}
\item You do not own a glider, flying with all the toys of your club. Not to 
get tired of changing polar and weight every time you change glider but XCSoar 
hardware via setup dialogues, set up different configuration files.
\item You got engaged in a contest. With difference in location and glider and 
having setup numerous screen pages for content purposes you want go back to 
your all day configuration quickly.
\item You are a flight instructor, handing out your glide computer once in a 
while in order to show your students what it is all about. Therefore you might 
need a much less complex configuration than you use normally.
\end{compactitem}
To backup configurations, make two copies of your profile.  One 
keeping extension \verb+.prf+, another one, changing the extension to
\verb+.bak+. Relating to the last example you might end up with
\begin{compactitem}
\item\verb+educationinASK13.prf+
\item\verb+educationinASK13.bak+
\item\verb+humphreyinLS8atwonderland.prf+
\item\verb+humphreyinLS8atwonderland.bak+
\end{compactitem}
Whenever XCSoar loads a specific profile \verb+.prf+, that file 
will include all configuration setups you will perform, with this file active. 
That is why a backup-file \verb+.bak+ is needed to preserve a standard, you 
judged being worth it.

\subsection{\textcolor{flashblue}{Mount Hardware}}
Mounting the hardware of your choice might turn out to reach some level of 
effort.  In any case, mount the hardware to even resist very rough conditions, 
no matter, whether you use special holders and a suction mount or a 
screwed-to-some-whatever.
In case, you just want to give it a try, holding your phone in your hands is a very 
bad idea. For this case, you might follow the hook-and-loop fastener way. A 
quick and dirty approach might be:
\begin{compactitem}
\item Get a silicone-type phone case, apply hook tape to the back.
\item Get a Velcro strap for wrapping around your leg, slightly above knee.
\item Put phone case including phone on your leg just after boarding, in order 
to achieve a knee board like operation.
\end{compactitem}
Never hold complex machines in your hand whilst flying.

%%%%%--------------------------------------------------------------------%%%%%
%%%%%--------------------------------------------------------------------%%%%%
%%%%%--------------------------------------------------------------------%%%%%
%%%%%--------------------------------------------------------------------%%%%%
%%%%%--------------------------------------------------------------------%%%%%
\newpage
\section{Pre-flight Check}
%%%%%--------------------------------------------------------------------%%%%%

%\begin{wrapfigure}[4]{R}{0.5\textwidth}
%\vspace{-4.5em}
%  \begin{flushright}
%    \includegraphics[width=0.48\textwidth]{figures/xy03.jpg}
%    xy preflight check
%  \end{flushright}
%\end{wrapfigure}

To be included in your Checklist.

If you use a single XCSoar device within several gliders, first setup your
plane's polar. Otherwise you might proceed with Setup Flight.

\subsection*{\textcolor{flashblue}{Setup Polar}}
Invoke either
\begin{flushleft}\hspace*{1cm}\config{2}{Plane}{}\hspace{1em} or\\
\end{flushleft}

\subsection*{\textcolor{flashblue}{Recall Setup}}
Restart XCSoar and choose a different configuration, including correctly set 
up polar and weight.

\subsection*{\textcolor{flashblue}{Setup Flight}}
\begin{flushleft}\hspace*{1cm}\config{1}{Flight}{}\blink\bmenuw{Wing loading / 
Bugs / QNH / Max temp}\\\end{flushleft}

\subsection*{\textcolor{flashblue}{Setup Wind}}
\begin{flushleft}\hspace*{1cm}\config{1}{Wind}{}\blink\bmenuw{Auto wind\dots / 
Speed / Direction}\\\end{flushleft}

\subsection*{\textcolor{flashblue}{Setup Task?}}
After having made your ``home'', XCSoar will always give you support on how to
come home, unless you have created a task. As you might have expected, XCSoar 
will guide you through your task and home from that time on.
For creating and using tasks, a brief introduction is given at the end of this 
flash. Set up either a ``Simple'' task or a``True'' task.


%%%%%--------------------------------------------------------------------%%%%%
%%%%%--------------------------------------------------------------------%%%%%
%%%%%--------------------------------------------------------------------%%%%%
%%%%%--------------------------------------------------------------------%%%%%
%%%%%--------------------------------------------------------------------%%%%%
\newpage
\section{Start Check}
%%%%%--------------------------------------------------------------------%%%%%

%\begin{wrapfigure}[4]{R}{0.5\textwidth}
%\vspace{-4.5em}
%  \begin{flushright}
%    \includegraphics[width=0.48\textwidth]{figures/xy04.jpg}
%    xy checking wind
%  \end{flushright}
%\end{wrapfigure}

To be included in your Checklist.

\subsection{\textcolor{flashblue}{Check Flight}}
\begin{flushleft}\hspace*{1cm}\config{1}{Flight}{}
%\blink\bmenuw{Wing loading / Bugs / QNH / Max temp}
\\\end{flushleft}

\subsection{\textcolor{flashblue}{Check Wind}}
\begin{flushleft}\hspace*{1cm}\config{1}{Wind}{}
%\blink\bmenuw{Auto wind\dots / Speed / Direction}
\\\end{flushleft}

\vspace{5em}
\section*{Fly}
During flight, there are two particular sets of info screens you might gather 
data from.
\subsection*{\textcolor{flashblue}{Flight Status}}
\begin{flushleft}\hspace*{1cm}\info{2}{Status}{}\blink\bmenuw{Times\dots}\\\end{flushleft}
\subsection*{\textcolor{flashblue}{Flight Analysis}}
\begin{flushleft}\hspace*{1cm}\info{1}{Analysis}{}\bmenuw{\blink}\dots\\\end{flushleft}

%%%%%--------------------------------------------------------------------%%%%%
%%%%%--------------------------------------------------------------------%%%%%
%%%%%--------------------------------------------------------------------%%%%%
%%%%%--------------------------------------------------------------------%%%%%
%%%%%--------------------------------------------------------------------%%%%%
\newpage
\section{After Flight Check}
%%%%%--------------------------------------------------------------------%%%%%

\subsection{\textcolor{flashblue}{Download Flight Log}}
Download flight logs from your preferred \emph{connected} logger.
\begin{flushleft}\hspace*{1cm}\config{2}{Devices}{}\\\end{flushleft}
\begin{compactitem}
\item select port
\item push \bmenuw{Flight download}
\item A listing of flightlogs stored on your external logger appears.
Choose one
\item push \bmenuw{Select}
\item wait....\ this is an old fashioned serial line working
\item choose another or push \bmenuw{No} to resume
\item depending on logger, once more wait patiently. Logger possibly will reset
and take its time to resume to normal operation.
\end{compactitem}

%\begin{wrapfigure}[8]{R}{0.4\textwidth}
%\vspace{-3em}
%  \begin{flushright}
%    \includegraphics[width=0.39\textwidth]{figures/xy05.jpg}
%    xy analyzing
%  \end{flushright}
%\end{wrapfigure}

\vspace{2em}
\textbf{Please note:} In case you are facing difficulties, make sure, the driver 
you are using supports downloads from logger. In case, you are using a cascaded 
system e.g. FLARM-Vario-XCSoar, it might be necessary to enable direct link or
similar on vario and choose another driver in XCSoar in order to gain access to 
downloads from FLARM to Xcsoar.

\subsection{\textcolor{flashblue}{Flight Analysis}}
\begin{flushleft}\hspace*{1cm}\info{1}{Analysis}{}\\\end{flushleft}
Use \bmenuw{\blink} to go through several screens showing statistical data 
gained during flight as are estimated OLC result, a barograph of your flight...
   ...and even a glide polar analysis if you had connected an instrument 
providing true airspeed measurement.

\subsection{\textcolor{flashblue}{Flight Status}}
\begin{flushleft}\hspace*{1cm}\info{2}{Status}{}\blink\bmenuw{Times\dots}\\\end{flushleft}
to gain more statistical information.

\subsection{\textcolor{flashblue}{Upload Flight}}
There are two portals waiting for your flight logs.
Skylines \url{https://skylines.aero} and
OLC \url{http://www.onlinecontest.org}
If you want to upload, presumably do it immediately after flight.  The OLC 
gives you a 24 hours time frame for uploading logs. Using Skylines is 
somewhat more relaxed and is your portal to online tracking and...
log files are in subdirectory \texttt{logs}.

\subsection*{\textcolor{flashblue}{Turn off Everything}}
Saving some current. Make sure, you have included \emph{everything}.

%%%%%--------------------------------------------------------------------%%%%%
%%%%%--------------------------------------------------------------------%%%%%
%%%%%--------------------------------------------------------------------%%%%%
%%%%%--------------------------------------------------------------------%%%%%
%%%%%--------------------------------------------------------------------%%%%%
\section{Tasks}
%%%%%--------------------------------------------------------------------%%%%%

\subsection{\textcolor{flashblue}{Retrieve ``home''-function, Goto-task}}
Since having defined a ``true'' task for the first time, the very basic navigate-
home function is shut down permanently. Use Goto function instead, providing
the most ``simple'' task. \textbf{Please note:} There are several functions as is 
InfoBoxes relating to the ``home'' waypoint. They do operate unaffected still. 
The only function shut down is just the \emph{surrogate} XCSoar uses with 
finding no user defined ``true'' task - or: until the user actively defines his 
very first task.
\subsection{\textcolor{flashblue}{Simple Task: Goto-task}}
There are several ways to create a simple task. In either case, choose a 
waypoint and push \bmenuw{Goto}.

Most common cases:
\begin{flushleft}\hspace*{1cm}\nav{1}{Waypoint}{List}\\\end{flushleft}
\begin{compactitem}
\item Use filtering menu screen to identify waypoint. Be aware, you can set 
multiple filtering. In case, no waypoint shows up, multiple filtering 
conditions might prevent from finding a target.
\item Push \bmenuw{OK}.
\item (select one) push \bmenuw{Select}, then \bmenuw{Goto}.
\end{compactitem}
Or use pan function of map, then proceed similar to case of ``making'' home.
\begin{compactitem}
\item Navigate to desired waypoint
\item Click/tap on (touch) screen in close proximity of desired waypoint.
\item Dialogue ``Map Elements at this location'' appears.
\item Choose waypoint.
\item Push \bmenuw{Goto}.
\end{compactitem}
 
\subsection{\textcolor{flashblue}{Task: create, save, load, declare}}
Creating a new task:
\begin{flushleft}\hspace*{1cm}\nav{1}{Task}{}\blink\bmenuw{Manage}\blink\bmenuw{New Task}\\\end{flushleft}
\begin{compactitem}
\item On the same menu screen you'll find \bmenuw{Browse}, \bmenuw{Save}, or 
\bmenuw{Declare} (to logger). After pushing \bmenuw{New Task} you are asked to
acknowledge, then menu screen for \bmenuw{Rules} comes up. Do your job, with 
defining the \bmenuw{Task type} first.
\item By touching \bmenuw{Turnpoint} and \bmenuw{Add Turnpoint} a filtering 
screen \bmenuw{Select Waypoint} comes up. Edit your wayponts. Touching 
\bmenuw{Edit Point} lets you alter things.
\item Do not close dialogue! Finally, use \bmenuw{Manage} and \bmenuw{Save} 
consecutively, then \bmenuw{Close}.
\end{compactitem}

\subsection*{\textcolor{flashblue}{Task: Start open time and close time}}
Right now, the flashing time initially set up is supposed to end.
Close time hit. Time to change to XCSoar's main manual.

Thank you for your time and

%\begin{wrapfigure}[4]{l}{0.5\textwidth}
%\vspace{-1.3em}
%  \begin{centering}
%    \includegraphics[width=0.48\textwidth]{figures/xy06.jpg}
%  \end{centering}
%\end{wrapfigure}

\section*{always happy landings}%no section at all, formatting reasons

%##############  Indexes of flash ##############
\input{toc.tex}


%##############  Closing page of flash ##############
\newpage
\pagestyle{empty}

\normalsize{for further information see:\\}
\vspace{0.5em}

\begin{center}
    \includegraphics[angle=0,width=0.5\textwidth,keepaspectratio='true']{graphics/logo.png}
    \vskip 0.5cm
%   option 1: web release 
%    \includegraphics[angle=0,width=0.66\textwidth,keepaspectratio='true']{graphics/title.pdf}\\
%   option 2: print release
    \fontsize{50}{0}
    \selectfont\textbf{XCSoar}\\
    \fontsize{12}{12}
    \vspace{0.2em}
    \LARGE{the open-source glide computer}\\
    \vspace{1.2em}
    \LARGE{User Manual}\\
    
\vspace{9em}

\end{center}
\begin{flushright}
\normalsize available at: \xcsoarwebsite{/discover/manual.html}\\
\end{flushright}

\end{document}
